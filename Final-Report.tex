% Options for packages loaded elsewhere
\PassOptionsToPackage{unicode}{hyperref}
\PassOptionsToPackage{hyphens}{url}
%
\documentclass[
]{article}
\usepackage{amsmath,amssymb}
\usepackage{iftex}
\ifPDFTeX
  \usepackage[T1]{fontenc}
  \usepackage[utf8]{inputenc}
  \usepackage{textcomp} % provide euro and other symbols
\else % if luatex or xetex
  \usepackage{unicode-math} % this also loads fontspec
  \defaultfontfeatures{Scale=MatchLowercase}
  \defaultfontfeatures[\rmfamily]{Ligatures=TeX,Scale=1}
\fi
\usepackage{lmodern}
\ifPDFTeX\else
  % xetex/luatex font selection
\fi
% Use upquote if available, for straight quotes in verbatim environments
\IfFileExists{upquote.sty}{\usepackage{upquote}}{}
\IfFileExists{microtype.sty}{% use microtype if available
  \usepackage[]{microtype}
  \UseMicrotypeSet[protrusion]{basicmath} % disable protrusion for tt fonts
}{}
\makeatletter
\@ifundefined{KOMAClassName}{% if non-KOMA class
  \IfFileExists{parskip.sty}{%
    \usepackage{parskip}
  }{% else
    \setlength{\parindent}{0pt}
    \setlength{\parskip}{6pt plus 2pt minus 1pt}}
}{% if KOMA class
  \KOMAoptions{parskip=half}}
\makeatother
\usepackage{xcolor}
\usepackage[margin=1in]{geometry}
\usepackage{color}
\usepackage{fancyvrb}
\newcommand{\VerbBar}{|}
\newcommand{\VERB}{\Verb[commandchars=\\\{\}]}
\DefineVerbatimEnvironment{Highlighting}{Verbatim}{commandchars=\\\{\}}
% Add ',fontsize=\small' for more characters per line
\usepackage{framed}
\definecolor{shadecolor}{RGB}{248,248,248}
\newenvironment{Shaded}{\begin{snugshade}}{\end{snugshade}}
\newcommand{\AlertTok}[1]{\textcolor[rgb]{0.94,0.16,0.16}{#1}}
\newcommand{\AnnotationTok}[1]{\textcolor[rgb]{0.56,0.35,0.01}{\textbf{\textit{#1}}}}
\newcommand{\AttributeTok}[1]{\textcolor[rgb]{0.13,0.29,0.53}{#1}}
\newcommand{\BaseNTok}[1]{\textcolor[rgb]{0.00,0.00,0.81}{#1}}
\newcommand{\BuiltInTok}[1]{#1}
\newcommand{\CharTok}[1]{\textcolor[rgb]{0.31,0.60,0.02}{#1}}
\newcommand{\CommentTok}[1]{\textcolor[rgb]{0.56,0.35,0.01}{\textit{#1}}}
\newcommand{\CommentVarTok}[1]{\textcolor[rgb]{0.56,0.35,0.01}{\textbf{\textit{#1}}}}
\newcommand{\ConstantTok}[1]{\textcolor[rgb]{0.56,0.35,0.01}{#1}}
\newcommand{\ControlFlowTok}[1]{\textcolor[rgb]{0.13,0.29,0.53}{\textbf{#1}}}
\newcommand{\DataTypeTok}[1]{\textcolor[rgb]{0.13,0.29,0.53}{#1}}
\newcommand{\DecValTok}[1]{\textcolor[rgb]{0.00,0.00,0.81}{#1}}
\newcommand{\DocumentationTok}[1]{\textcolor[rgb]{0.56,0.35,0.01}{\textbf{\textit{#1}}}}
\newcommand{\ErrorTok}[1]{\textcolor[rgb]{0.64,0.00,0.00}{\textbf{#1}}}
\newcommand{\ExtensionTok}[1]{#1}
\newcommand{\FloatTok}[1]{\textcolor[rgb]{0.00,0.00,0.81}{#1}}
\newcommand{\FunctionTok}[1]{\textcolor[rgb]{0.13,0.29,0.53}{\textbf{#1}}}
\newcommand{\ImportTok}[1]{#1}
\newcommand{\InformationTok}[1]{\textcolor[rgb]{0.56,0.35,0.01}{\textbf{\textit{#1}}}}
\newcommand{\KeywordTok}[1]{\textcolor[rgb]{0.13,0.29,0.53}{\textbf{#1}}}
\newcommand{\NormalTok}[1]{#1}
\newcommand{\OperatorTok}[1]{\textcolor[rgb]{0.81,0.36,0.00}{\textbf{#1}}}
\newcommand{\OtherTok}[1]{\textcolor[rgb]{0.56,0.35,0.01}{#1}}
\newcommand{\PreprocessorTok}[1]{\textcolor[rgb]{0.56,0.35,0.01}{\textit{#1}}}
\newcommand{\RegionMarkerTok}[1]{#1}
\newcommand{\SpecialCharTok}[1]{\textcolor[rgb]{0.81,0.36,0.00}{\textbf{#1}}}
\newcommand{\SpecialStringTok}[1]{\textcolor[rgb]{0.31,0.60,0.02}{#1}}
\newcommand{\StringTok}[1]{\textcolor[rgb]{0.31,0.60,0.02}{#1}}
\newcommand{\VariableTok}[1]{\textcolor[rgb]{0.00,0.00,0.00}{#1}}
\newcommand{\VerbatimStringTok}[1]{\textcolor[rgb]{0.31,0.60,0.02}{#1}}
\newcommand{\WarningTok}[1]{\textcolor[rgb]{0.56,0.35,0.01}{\textbf{\textit{#1}}}}
\usepackage{longtable,booktabs,array}
\usepackage{calc} % for calculating minipage widths
% Correct order of tables after \paragraph or \subparagraph
\usepackage{etoolbox}
\makeatletter
\patchcmd\longtable{\par}{\if@noskipsec\mbox{}\fi\par}{}{}
\makeatother
% Allow footnotes in longtable head/foot
\IfFileExists{footnotehyper.sty}{\usepackage{footnotehyper}}{\usepackage{footnote}}
\makesavenoteenv{longtable}
\usepackage{graphicx}
\makeatletter
\def\maxwidth{\ifdim\Gin@nat@width>\linewidth\linewidth\else\Gin@nat@width\fi}
\def\maxheight{\ifdim\Gin@nat@height>\textheight\textheight\else\Gin@nat@height\fi}
\makeatother
% Scale images if necessary, so that they will not overflow the page
% margins by default, and it is still possible to overwrite the defaults
% using explicit options in \includegraphics[width, height, ...]{}
\setkeys{Gin}{width=\maxwidth,height=\maxheight,keepaspectratio}
% Set default figure placement to htbp
\makeatletter
\def\fps@figure{htbp}
\makeatother
\setlength{\emergencystretch}{3em} % prevent overfull lines
\providecommand{\tightlist}{%
  \setlength{\itemsep}{0pt}\setlength{\parskip}{0pt}}
\setcounter{secnumdepth}{-\maxdimen} % remove section numbering
\ifLuaTeX
  \usepackage{selnolig}  % disable illegal ligatures
\fi
\usepackage{bookmark}
\IfFileExists{xurl.sty}{\usepackage{xurl}}{} % add URL line breaks if available
\urlstyle{same}
\hypersetup{
  pdftitle={Final Report},
  pdfauthor={Liz Kim},
  hidelinks,
  pdfcreator={LaTeX via pandoc}}

\title{Final Report}
\author{Liz Kim}
\date{1/21/2025}

\begin{document}
\maketitle

\subsection{Table of Contents}\label{table-of-contents}

\begin{itemize}
\tightlist
\item
  \hyperref[introduction]{1. Introduction}
\item
  \hyperref[research-question]{2. Research Question}
\item
  \hyperref[Data-Scaping-and-Tidying-the-Data]{3. Data Scraping \&
  Tidying the Data}
\item
  \hyperref[descriptive-statistics-and-plots]{4. Descriptive Statistics
  and Plots}
\item
  \hyperref[data-interpretation]{5. Data Interpretation}
\end{itemize}

\section{1. Introduction}\label{introduction}

This project investigates the factors associated with national
happiness, utilizing happiness scores from the World Population Review
and key factors from Wikipedia tables. Key variables include GDP per
capita, environmental quality, abd literacy rates offering insights into
the economic, environmental, and educational factors that influence
happiness levels across countries.

The reason why I'm working on this project is because it is important to
understand how these factors contribute to national happiness. Happiness
is a critical measure of well-being, and identifying its determinants is
vital.

\section{2. Research Question}\label{research-question}

Main Question: How do economic,education, and environmental variables
influence national happiness levels as reported by the World Population
Review?

RQ1: How does GDP per capita associate with a country's happiness score?
RQ2: How does environmental quality associate with a country's happiness
score? RQ3: How does a country's literacy rate associate with its
happiness score?

\section{3. Data Scraping \& Tidying the
Data}\label{data-scraping-tidying-the-data}

Below is the code I used to scrap the data.

\textbf{Happiness Score}

\begin{Shaded}
\begin{Highlighting}[]
\CommentTok{\# Read the CSV file}
\NormalTok{happiest\_countries }\OtherTok{\textless{}{-}} \FunctionTok{read\_csv}\NormalTok{(}\StringTok{"happiest{-}countries{-}in{-}the{-}world{-}2024.csv"}\NormalTok{)}
\end{Highlighting}
\end{Shaded}

\begin{verbatim}
## Rows: 145 Columns: 8
## -- Column specification --------------------------------------------------------
## Delimiter: ","
## chr (1): country
## dbl (7): HappiestCountriesWorldHappinessReportRankings2024, HappiestCountrie...
## 
## i Use `spec()` to retrieve the full column specification for this data.
## i Specify the column types or set `show_col_types = FALSE` to quiet this message.
\end{verbatim}

\begin{Shaded}
\begin{Highlighting}[]
\CommentTok{\# Preview the data}
\FunctionTok{head}\NormalTok{(happiest\_countries)}
\end{Highlighting}
\end{Shaded}

\begin{verbatim}
## # A tibble: 6 x 8
##   country       HappiestCountriesWorl~1 HappiestCountriesWor~2 LatestScoreChange
##   <chr>                           <dbl>                  <dbl>             <dbl>
## 1 India                             126                   4.05            0.0140
## 2 China                              60                   5.97            0.152 
## 3 United States                      23                   6.73           -0.164 
## 4 Indonesia                          80                   5.57            0.293 
## 5 Pakistan                          108                   4.66            0.105 
## 6 Nigeria                           102                   4.88           -0.101 
## # i abbreviated names: 1: HappiestCountriesWorldHappinessReportRankings2024,
## #   2: HappiestCountriesWorldHappinessReportScore2024
## # i 4 more variables: HappiestCountriesWorldHappiessReportRankings2023 <dbl>,
## #   HappiestCountriesWorldHappiessReportScore2023 <dbl>,
## #   HappiestCountriesWorldHappiessReportRankings2022 <dbl>,
## #   HappiestCountriesWorldHappiessReportScore2022 <dbl>
\end{verbatim}

I requested this data table from WorldPopulationReview.com. I received
an excel file through my email.

\textbf{GDP per Capita}

\begin{Shaded}
\begin{Highlighting}[]
\CommentTok{\# Save the URL and scrape the webpage}
\NormalTok{url }\OtherTok{\textless{}{-}} \StringTok{"https://en.wikipedia.org/wiki/List\_of\_countries\_by\_GDP\_(nominal)\_per\_capita"}
\NormalTok{gdp\_page }\OtherTok{\textless{}{-}} \FunctionTok{read\_html}\NormalTok{(}\AttributeTok{x =}\NormalTok{ url)}

\CommentTok{\# Extract all tables with the class "wikitable"}
\NormalTok{tables }\OtherTok{\textless{}{-}} \FunctionTok{html\_elements}\NormalTok{(gdp\_page, }\AttributeTok{css =} \StringTok{"table.wikitable"}\NormalTok{)}

\CommentTok{\# Convert the first table to a data frame}
\NormalTok{gdp\_per\_capita\_table }\OtherTok{\textless{}{-}} \FunctionTok{html\_table}\NormalTok{(tables[[}\DecValTok{1}\NormalTok{]], }\AttributeTok{fill =} \ConstantTok{TRUE}\NormalTok{)}

\CommentTok{\# Save the table as a CSV file}
\FunctionTok{write\_csv}\NormalTok{(gdp\_per\_capita\_table, }\StringTok{"gdp\_per\_capita.csv"}\NormalTok{)}

\CommentTok{\# Preview the table}
\FunctionTok{head}\NormalTok{(gdp\_per\_capita\_table)}
\end{Highlighting}
\end{Shaded}

\begin{verbatim}
## # A tibble: 6 x 7
##   `Country/Territory` `IMF[4][5]` `IMF[4][5]` `World Bank[6]` `World Bank[6]`
##   <chr>               <chr>       <chr>       <chr>           <chr>          
## 1 Country/Territory   Estimate    Year        Estimate        Year           
## 2 Monaco              —           —           240,862         2022           
## 3 Liechtenstein       —           —           187,267         2022           
## 4 Luxembourg          135,321     2024        128,259         2023           
## 5 Bermuda             —           —           123,091         2022           
## 6 Switzerland         106,098     2024        99,995          2023           
## # i 2 more variables: `United Nations[7]` <chr>, `United Nations[7]` <chr>
\end{verbatim}

\textbf{Education (Literacy Rate)}

\begin{Shaded}
\begin{Highlighting}[]
\CommentTok{\# Save the URL and scrape the webpage}
\NormalTok{url }\OtherTok{\textless{}{-}} \StringTok{"https://en.wikipedia.org/wiki/List\_of\_countries\_by\_literacy\_rate"}
\NormalTok{literacy\_page }\OtherTok{\textless{}{-}} \FunctionTok{read\_html}\NormalTok{(}\AttributeTok{x =}\NormalTok{ url)}

\CommentTok{\# Extract all tables with the class "wikitable"}
\NormalTok{tables }\OtherTok{\textless{}{-}} \FunctionTok{html\_elements}\NormalTok{(literacy\_page, }\AttributeTok{css =} \StringTok{"table.wikitable"}\NormalTok{)}

\CommentTok{\# Convert all table nodes into a list of data frames}
\NormalTok{all\_tables }\OtherTok{\textless{}{-}} \FunctionTok{html\_table}\NormalTok{(tables, }\AttributeTok{fill =} \ConstantTok{TRUE}\NormalTok{)}

\CommentTok{\# Inspect tables to find the one I want to extract}
\ControlFlowTok{for}\NormalTok{ (i }\ControlFlowTok{in} \FunctionTok{seq\_along}\NormalTok{(all\_tables)) \{}
  \FunctionTok{print}\NormalTok{(}\FunctionTok{paste}\NormalTok{(}\StringTok{"Table"}\NormalTok{, i))}
  \FunctionTok{print}\NormalTok{(}\FunctionTok{head}\NormalTok{(all\_tables[[i]]))}
\NormalTok{\}}
\end{Highlighting}
\end{Shaded}

\begin{verbatim}
## [1] "Table 1"
## # A tibble: 6 x 9
##   Country          `Youth(15 to 24)` `Youth(15 to 24)` `Adult(25+)` `Adult(25+)`
##   <chr>            <chr>             <chr>             <chr>        <chr>       
## 1 Country          "Rate"            "Year"            "Rate"       "Year"      
## 2 Afghanistan *    "65.0"            "2020[4]"         "31.7"       "2011"      
## 3 Albania *        "99.2"            "2012"            "97.2"       "2012"      
## 4 Algeria *        "93.8"            "2008"            "75.1"       "2008"      
## 5 American Samoa * "97.7"            "1980"            "97.3"       "1980"      
## 6 Andorra *        ""                ""                ""           ""          
## # i 4 more variables: `Elderly(65+)` <chr>, `Elderly(65+)` <chr>,
## #   `Youth GenderParity Index` <chr>, `Youth GenderParity Index` <chr>
## [1] "Table 2"
## # A tibble: 6 x 6
##   Country   Literacy rate[12][13~1 Literacy rate[12][13~2 Literacy rate[12][13~3
##   <chr>     <chr>                  <chr>                  <chr>                 
## 1 Country   Total                  Male                   Female                
## 2 Afghanis~ 37.3%                  52.1%                  22.6%                 
## 3 Albania * 98.1%                  98.5%                  97.8%                 
## 4 Algeria * 81.4%                  87.4%                  75.3%                 
## 5 Andorra * 100.0%                 100.0%                 100.0%                
## 6 Angola *  71.1%                  82.0%                  60.7%                 
## # i abbreviated names: 1: `Literacy rate[12][13][text–source integrity?]`,
## #   2: `Literacy rate[12][13][text–source integrity?]`,
## #   3: `Literacy rate[12][13][text–source integrity?]`
## # i 2 more variables: `Literacy rate[12][13][text–source integrity?]` <chr>,
## #   `Literacy rate[12][13][text–source integrity?]` <chr>
## [1] "Table 3"
## # A tibble: 6 x 7
##   Territory      `Literacy rate (all)` `Male literacy` `Female literacy`
##   <chr>          <chr>                 <chr>           <chr>            
## 1 Aruba          97.5%                 97.5%           97.5%            
## 2 Cayman Islands 98.9%                 98.7%           99.0%            
## 3 Guadeloupe     96.5%                 96.4%           96.6%            
## 4 Guam           99.8%                 99.8%           99.8%            
## 5 Kosovo         91.9%                 96.6%           87.5%            
## 6 Macau          96.2%                 98.0%           94.6%            
## # i 3 more variables: `Gender difference[a]` <chr>, Year <chr>, Note <chr>
\end{verbatim}

\begin{Shaded}
\begin{Highlighting}[]
\CommentTok{\# Extract the second table }
\NormalTok{literacy\_rate\_table }\OtherTok{\textless{}{-}}\NormalTok{ all\_tables[[}\DecValTok{2}\NormalTok{]]}
\FunctionTok{write\_csv}\NormalTok{(literacy\_rate\_table, }\StringTok{"literacy\_rate.csv"}\NormalTok{)}

\CommentTok{\# Preview the table}
\FunctionTok{head}\NormalTok{(literacy\_rate\_table)}
\end{Highlighting}
\end{Shaded}

\begin{verbatim}
## # A tibble: 6 x 6
##   Country   Literacy rate[12][13~1 Literacy rate[12][13~2 Literacy rate[12][13~3
##   <chr>     <chr>                  <chr>                  <chr>                 
## 1 Country   Total                  Male                   Female                
## 2 Afghanis~ 37.3%                  52.1%                  22.6%                 
## 3 Albania * 98.1%                  98.5%                  97.8%                 
## 4 Algeria * 81.4%                  87.4%                  75.3%                 
## 5 Andorra * 100.0%                 100.0%                 100.0%                
## 6 Angola *  71.1%                  82.0%                  60.7%                 
## # i abbreviated names: 1: `Literacy rate[12][13][text–source integrity?]`,
## #   2: `Literacy rate[12][13][text–source integrity?]`,
## #   3: `Literacy rate[12][13][text–source integrity?]`
## # i 2 more variables: `Literacy rate[12][13][text–source integrity?]` <chr>,
## #   `Literacy rate[12][13][text–source integrity?]` <chr>
\end{verbatim}

\textbf{Environmental Quality}

\begin{Shaded}
\begin{Highlighting}[]
\CommentTok{\# Save the URL and scrape the webpage}
\NormalTok{url }\OtherTok{\textless{}{-}} \StringTok{"https://en.wikipedia.org/wiki/Environmental\_Performance\_Index"}
\NormalTok{environment\_page }\OtherTok{\textless{}{-}} \FunctionTok{read\_html}\NormalTok{(}\AttributeTok{x =}\NormalTok{ url)}

\CommentTok{\# Extract all tables with the class "wikitable"}
\NormalTok{tables }\OtherTok{\textless{}{-}} \FunctionTok{html\_elements}\NormalTok{(environment\_page, }\AttributeTok{css =} \StringTok{"table.wikitable"}\NormalTok{)}

\CommentTok{\# Convert all table nodes into a list of data frames}
\NormalTok{all\_tables }\OtherTok{\textless{}{-}} \FunctionTok{html\_table}\NormalTok{(tables, }\AttributeTok{fill =} \ConstantTok{TRUE}\NormalTok{)}

\CommentTok{\# Inspect tables to find the one I want to extract}
\ControlFlowTok{for}\NormalTok{ (i }\ControlFlowTok{in} \FunctionTok{seq\_along}\NormalTok{(all\_tables)) \{}
  \FunctionTok{print}\NormalTok{(}\FunctionTok{paste}\NormalTok{(}\StringTok{"Table"}\NormalTok{, i))}
  \FunctionTok{print}\NormalTok{(}\FunctionTok{head}\NormalTok{(all\_tables[[i]]))}
\NormalTok{\}}
\end{Highlighting}
\end{Shaded}

\begin{verbatim}
## [1] "Table 1"
## # A tibble: 6 x 6
##   `Policy Objective` `Wt. (%)` `Issue Category`    `Wt. (%)` Indicator `Wt. (%)`
##   <chr>              <chr>     <chr>               <chr>     <chr>     <chr>    
## 1 Ecosystem Vitality 45%       Biodiversity & Hab~ 25        Marine K~ 12.0     
## 2 Ecosystem Vitality 45%       Biodiversity & Hab~ 25        Marine H~ 12.0     
## 3 Ecosystem Vitality 45%       Biodiversity & Hab~ 25        Marine P~ 2.0      
## 4 Ecosystem Vitality 45%       Biodiversity & Hab~ 25        Protecte~ 12.0     
## 5 Ecosystem Vitality 45%       Biodiversity & Hab~ 25        Species ~ 16.0     
## 6 Ecosystem Vitality 45%       Biodiversity & Hab~ 25        Terrestr~ 10.0     
## [1] "Table 2"
## # A tibble: 6 x 5
##   Country              Region                     Value Trend `Rank 2024`
##   <chr>                <chr>                      <dbl> <chr>       <int>
## 1 Afghanistan          South Asia                  31   12.8          144
## 2 Angola               Sub-Saharan Africa          40.1 8.2           106
## 3 Albania              Europe & Central Asia       52.2 6.1            47
## 4 United Arab Emirates Middle East & North Africa  51.6 9.1            48
## 5 Argentina            Latin America & Caribbean   47   1.1            70
## 6 Armenia              Europe & Central Asia       44.9 2.0            80
\end{verbatim}

\begin{Shaded}
\begin{Highlighting}[]
\CommentTok{\# Extract the second table (Environmental Performance Index)}
\NormalTok{environmental\_quality\_table }\OtherTok{\textless{}{-}}\NormalTok{ all\_tables[[}\DecValTok{2}\NormalTok{]]}
\FunctionTok{write\_csv}\NormalTok{(environmental\_quality\_table, }\StringTok{"environmental\_quality.csv"}\NormalTok{)}

\CommentTok{\# Preview the table}
\FunctionTok{head}\NormalTok{(environmental\_quality\_table)}
\end{Highlighting}
\end{Shaded}

\begin{verbatim}
## # A tibble: 6 x 5
##   Country              Region                     Value Trend `Rank 2024`
##   <chr>                <chr>                      <dbl> <chr>       <int>
## 1 Afghanistan          South Asia                  31   12.8          144
## 2 Angola               Sub-Saharan Africa          40.1 8.2           106
## 3 Albania              Europe & Central Asia       52.2 6.1            47
## 4 United Arab Emirates Middle East & North Africa  51.6 9.1            48
## 5 Argentina            Latin America & Caribbean   47   1.1            70
## 6 Armenia              Europe & Central Asia       44.9 2.0            80
\end{verbatim}

Here, I included the code to tidy the tables. The datasets comprises
four tables, each offering insights into countries' happiness scores,
GDP, literacy rates, and environmental performance. To briefly define
``tidy'' data and explain its advantages and disadvantages,

\subsection{Happiness Score Table}\label{happiness-score-table}

\begin{Shaded}
\begin{Highlighting}[]
\CommentTok{\# Read the previously saved CSV file}
\NormalTok{happiest\_countries }\OtherTok{\textless{}{-}} \FunctionTok{read\_csv}\NormalTok{(}\StringTok{"happiest{-}countries{-}in{-}the{-}world{-}2024.csv"}\NormalTok{)}
\end{Highlighting}
\end{Shaded}

\begin{verbatim}
## Rows: 145 Columns: 8
## -- Column specification --------------------------------------------------------
## Delimiter: ","
## chr (1): country
## dbl (7): HappiestCountriesWorldHappinessReportRankings2024, HappiestCountrie...
## 
## i Use `spec()` to retrieve the full column specification for this data.
## i Specify the column types or set `show_col_types = FALSE` to quiet this message.
\end{verbatim}

\begin{Shaded}
\begin{Highlighting}[]
\CommentTok{\# Clean and rename the happiest\_countries dataset}
\NormalTok{happiest\_countries\_cleaned }\OtherTok{\textless{}{-}}\NormalTok{ happiest\_countries }\SpecialCharTok{\%\textgreater{}\%}
  \FunctionTok{select}\NormalTok{(country, HappiestCountriesWorldHappinessReportScore2024) }\SpecialCharTok{\%\textgreater{}\%} 
  \FunctionTok{filter}\NormalTok{(}\SpecialCharTok{!}\FunctionTok{is.na}\NormalTok{(HappiestCountriesWorldHappinessReportScore2024)) }\SpecialCharTok{\%\textgreater{}\%} 
  \FunctionTok{rename}\NormalTok{(}\AttributeTok{HappinessScore =}\NormalTok{ HappiestCountriesWorldHappinessReportScore2024) }

\CommentTok{\# Preview and save the cleaned table}
\FunctionTok{head}\NormalTok{(happiest\_countries\_cleaned)}
\end{Highlighting}
\end{Shaded}

\begin{verbatim}
## # A tibble: 6 x 2
##   country       HappinessScore
##   <chr>                  <dbl>
## 1 India                   4.05
## 2 China                   5.97
## 3 United States           6.73
## 4 Indonesia               5.57
## 5 Pakistan                4.66
## 6 Nigeria                 4.88
\end{verbatim}

\begin{Shaded}
\begin{Highlighting}[]
\FunctionTok{write\_csv}\NormalTok{(happiest\_countries\_cleaned, }\StringTok{"happiest\_countries\_cleaned.csv"}\NormalTok{)}
\end{Highlighting}
\end{Shaded}

\subsection{GDP Table}\label{gdp-table}

\begin{Shaded}
\begin{Highlighting}[]
\CommentTok{\# Read the previously saved CSV file}
\NormalTok{gdp\_per\_capita\_table }\OtherTok{\textless{}{-}} \FunctionTok{read\_csv}\NormalTok{(}\StringTok{"gdp\_per\_capita.csv"}\NormalTok{)}
\end{Highlighting}
\end{Shaded}

\begin{verbatim}
## New names:
## Rows: 224 Columns: 7
## -- Column specification
## -------------------------------------------------------- Delimiter: "," chr
## (7): Country/Territory, IMF[4][5]...2, IMF[4][5]...3, World Bank[6]...4,...
## i Use `spec()` to retrieve the full column specification for this data. i
## Specify the column types or set `show_col_types = FALSE` to quiet this message.
## * `IMF[4][5]` -> `IMF[4][5]...2`
## * `IMF[4][5]` -> `IMF[4][5]...3`
## * `World Bank[6]` -> `World Bank[6]...4`
## * `World Bank[6]` -> `World Bank[6]...5`
## * `United Nations[7]` -> `United Nations[7]...6`
## * `United Nations[7]` -> `United Nations[7]...7`
\end{verbatim}

\begin{Shaded}
\begin{Highlighting}[]
\CommentTok{\# Rename columns}
\FunctionTok{colnames}\NormalTok{(gdp\_per\_capita\_table) }\OtherTok{\textless{}{-}} \FunctionTok{c}\NormalTok{(}\StringTok{"country"}\NormalTok{, }\StringTok{"IMF\_estimate"}\NormalTok{, }\StringTok{"IMF\_year"}\NormalTok{, }
                                    \StringTok{"World\_Bank\_estimate"}\NormalTok{, }\StringTok{"World\_Bank\_year"}\NormalTok{, }
                                    \StringTok{"UN\_estimate"}\NormalTok{, }\StringTok{"UN\_year"}\NormalTok{)}

\CommentTok{\# Tidy the table}
\NormalTok{fill\_missing\_values }\OtherTok{\textless{}{-}} \ControlFlowTok{function}\NormalTok{(data) \{}
\NormalTok{  data }\SpecialCharTok{\%\textgreater{}\%}
    \FunctionTok{select}\NormalTok{(country, IMF\_estimate) }\SpecialCharTok{\%\textgreater{}\%}              
    \FunctionTok{mutate}\NormalTok{(}
      \AttributeTok{IMF\_estimate =} \FunctionTok{ifelse}\NormalTok{(IMF\_estimate }\SpecialCharTok{==} \StringTok{"{-}"}\NormalTok{, }\ConstantTok{NA}\NormalTok{, IMF\_estimate), }
      \AttributeTok{IMF\_estimate =} \FunctionTok{gsub}\NormalTok{(}\StringTok{","}\NormalTok{, }\StringTok{""}\NormalTok{, IMF\_estimate),      }
      \AttributeTok{IMF\_estimate =} \FunctionTok{as.numeric}\NormalTok{(IMF\_estimate)         }
\NormalTok{    ) }\SpecialCharTok{\%\textgreater{}\%}
    \FunctionTok{fill}\NormalTok{(IMF\_estimate, }\AttributeTok{.direction =} \StringTok{"down"}\NormalTok{) }\SpecialCharTok{\%\textgreater{}\%}       
    \FunctionTok{filter}\NormalTok{(}\SpecialCharTok{!}\FunctionTok{is.na}\NormalTok{(IMF\_estimate)) }\SpecialCharTok{\%\textgreater{}\%}                  
    \FunctionTok{rename}\NormalTok{(}\AttributeTok{estimate =}\NormalTok{ IMF\_estimate)                  }
\NormalTok{\}}

\CommentTok{\# Apply the cleaning function}
\NormalTok{gdp\_per\_capita\_table\_cleaned }\OtherTok{\textless{}{-}} \FunctionTok{fill\_missing\_values}\NormalTok{(gdp\_per\_capita\_table)}
\end{Highlighting}
\end{Shaded}

\begin{verbatim}
## Warning: There was 1 warning in `mutate()`.
## i In argument: `IMF_estimate = as.numeric(IMF_estimate)`.
## Caused by warning:
## ! NAs introduced by coercion
\end{verbatim}

\begin{Shaded}
\begin{Highlighting}[]
\CommentTok{\# Preview and save the cleaned table}
\FunctionTok{head}\NormalTok{(gdp\_per\_capita\_table\_cleaned) }
\end{Highlighting}
\end{Shaded}

\begin{verbatim}
## # A tibble: 6 x 2
##   country        estimate
##   <chr>             <dbl>
## 1 Luxembourg       135321
## 2 Bermuda          135321
## 3 Switzerland      106098
## 4 Ireland          103500
## 5 Cayman Islands   103500
## 6 Isle of Man      103500
\end{verbatim}

\begin{Shaded}
\begin{Highlighting}[]
\FunctionTok{write\_csv}\NormalTok{(gdp\_per\_capita\_table\_cleaned, }\StringTok{"gdp\_per\_capita\_table\_cleaned.csv"}\NormalTok{)}
\end{Highlighting}
\end{Shaded}

\subsection{Education (Literacy Rate
Table)}\label{education-literacy-rate-table}

\begin{Shaded}
\begin{Highlighting}[]
\CommentTok{\# Read the previously saved CSV file}
\NormalTok{literacy\_rate\_table }\OtherTok{\textless{}{-}} \FunctionTok{read\_csv}\NormalTok{(}\StringTok{"literacy\_rate.csv"}\NormalTok{)}
\end{Highlighting}
\end{Shaded}

\begin{verbatim}
## New names:
## Rows: 197 Columns: 6
## -- Column specification
## -------------------------------------------------------- Delimiter: "," chr
## (6): Country, Literacy rate[12][13][text–source integrity?]...2, Literac...
## i Use `spec()` to retrieve the full column specification for this data. i
## Specify the column types or set `show_col_types = FALSE` to quiet this message.
## * `Literacy rate[12][13][text–source integrity?]` -> `Literacy
##   rate[12][13][text–source integrity?]...2`
## * `Literacy rate[12][13][text–source integrity?]` -> `Literacy
##   rate[12][13][text–source integrity?]...3`
## * `Literacy rate[12][13][text–source integrity?]` -> `Literacy
##   rate[12][13][text–source integrity?]...4`
## * `Literacy rate[12][13][text–source integrity?]` -> `Literacy
##   rate[12][13][text–source integrity?]...5`
## * `Literacy rate[12][13][text–source integrity?]` -> `Literacy
##   rate[12][13][text–source integrity?]...6`
\end{verbatim}

\begin{Shaded}
\begin{Highlighting}[]
\FunctionTok{head}\NormalTok{(literacy\_rate\_table)}
\end{Highlighting}
\end{Shaded}

\begin{verbatim}
## # A tibble: 6 x 6
##   Country   Literacy rate[12][13~1 Literacy rate[12][13~2 Literacy rate[12][13~3
##   <chr>     <chr>                  <chr>                  <chr>                 
## 1 Country   Total                  Male                   Female                
## 2 Afghanis~ 37.3%                  52.1%                  22.6%                 
## 3 Albania * 98.1%                  98.5%                  97.8%                 
## 4 Algeria * 81.4%                  87.4%                  75.3%                 
## 5 Andorra * 100.0%                 100.0%                 100.0%                
## 6 Angola *  71.1%                  82.0%                  60.7%                 
## # i abbreviated names: 1: `Literacy rate[12][13][text–source integrity?]...2`,
## #   2: `Literacy rate[12][13][text–source integrity?]...3`,
## #   3: `Literacy rate[12][13][text–source integrity?]...4`
## # i 2 more variables:
## #   `Literacy rate[12][13][text–source integrity?]...5` <chr>,
## #   `Literacy rate[12][13][text–source integrity?]...6` <chr>
\end{verbatim}

\begin{Shaded}
\begin{Highlighting}[]
\CommentTok{\# Rename columns}
\FunctionTok{colnames}\NormalTok{(literacy\_rate\_table) }\OtherTok{\textless{}{-}} \FunctionTok{c}\NormalTok{(}\StringTok{"country"}\NormalTok{, }\StringTok{"total"}\NormalTok{, }\StringTok{"male"}\NormalTok{, }\StringTok{"female"}\NormalTok{, }\StringTok{"gap"}\NormalTok{, }\StringTok{"year"}\NormalTok{)}

\CommentTok{\# Tidy the data}
\NormalTok{literacy\_rate\_table\_cleaned }\OtherTok{\textless{}{-}}\NormalTok{ literacy\_rate\_table }\SpecialCharTok{\%\textgreater{}\%}
  \FunctionTok{select}\NormalTok{(country, total, year) }\SpecialCharTok{\%\textgreater{}\%}
  \FunctionTok{mutate}\NormalTok{(}
    \AttributeTok{country =} \FunctionTok{gsub}\NormalTok{(}\StringTok{"}\SpecialCharTok{\textbackslash{}\textbackslash{}}\StringTok{*"}\NormalTok{, }\StringTok{""}\NormalTok{, country),  }
    \AttributeTok{total =} \FunctionTok{as.numeric}\NormalTok{(}\FunctionTok{gsub}\NormalTok{(}\StringTok{"\%"}\NormalTok{, }\StringTok{""}\NormalTok{, total)), }
    \AttributeTok{year =} \FunctionTok{as.numeric}\NormalTok{(year)                 }
\NormalTok{  )}
\end{Highlighting}
\end{Shaded}

\begin{verbatim}
## Warning: There were 2 warnings in `mutate()`.
## The first warning was:
## i In argument: `total = as.numeric(gsub("%", "", total))`.
## Caused by warning:
## ! NAs introduced by coercion
## i Run `dplyr::last_dplyr_warnings()` to see the 1 remaining warning.
\end{verbatim}

\begin{Shaded}
\begin{Highlighting}[]
\CommentTok{\# Filter out rows where country == "Country" or "World"}
\NormalTok{literacy\_rate\_table\_cleaned }\OtherTok{\textless{}{-}}\NormalTok{ literacy\_rate\_table\_cleaned }\SpecialCharTok{\%\textgreater{}\%}
  \FunctionTok{filter}\NormalTok{(country }\SpecialCharTok{!=} \StringTok{"Country"}\NormalTok{,              }
\NormalTok{         country }\SpecialCharTok{!=} \StringTok{"World"}\NormalTok{)    }

\CommentTok{\# Preview and save the cleaned table}
\FunctionTok{head}\NormalTok{(literacy\_rate\_table\_cleaned)}
\end{Highlighting}
\end{Shaded}

\begin{verbatim}
## # A tibble: 6 x 3
##   country              total  year
##   <chr>                <dbl> <dbl>
## 1 Afghanistan           37.3  2021
## 2 Albania               98.1  2018
## 3 Algeria               81.4  2018
## 4 Andorra              100    2016
## 5 Angola                71.1  2015
## 6 Antigua and Barbuda   99    2015
\end{verbatim}

\begin{Shaded}
\begin{Highlighting}[]
\FunctionTok{write\_csv}\NormalTok{(literacy\_rate\_table\_cleaned, }\StringTok{"literacy\_rate\_table\_cleaned.csv"}\NormalTok{)}
\end{Highlighting}
\end{Shaded}

\subsection{Environmental Quality
Table}\label{environmental-quality-table}

\begin{Shaded}
\begin{Highlighting}[]
\CommentTok{\# Read the previously saved CSV file}
\NormalTok{environmental\_quality\_table }\OtherTok{\textless{}{-}} \FunctionTok{read\_csv}\NormalTok{(}\StringTok{"environmental\_quality.csv"}\NormalTok{)}
\end{Highlighting}
\end{Shaded}

\begin{verbatim}
## Rows: 179 Columns: 5
## -- Column specification --------------------------------------------------------
## Delimiter: ","
## chr (3): Country, Region, Trend
## dbl (2): Value, Rank 2024
## 
## i Use `spec()` to retrieve the full column specification for this data.
## i Specify the column types or set `show_col_types = FALSE` to quiet this message.
\end{verbatim}

\begin{Shaded}
\begin{Highlighting}[]
\CommentTok{\# Tidy the data}
\NormalTok{environmental\_quality\_table\_cleaned }\OtherTok{\textless{}{-}}\NormalTok{ environmental\_quality\_table }\SpecialCharTok{\%\textgreater{}\%}
  \FunctionTok{select}\NormalTok{(}\AttributeTok{country =} \StringTok{\textasciigrave{}}\AttributeTok{Country}\StringTok{\textasciigrave{}}\NormalTok{, }\AttributeTok{value =} \StringTok{\textasciigrave{}}\AttributeTok{Value}\StringTok{\textasciigrave{}}\NormalTok{)}

\CommentTok{\# Preview and save the cleaned table}
\FunctionTok{head}\NormalTok{(environmental\_quality\_table\_cleaned)}
\end{Highlighting}
\end{Shaded}

\begin{verbatim}
## # A tibble: 6 x 2
##   country              value
##   <chr>                <dbl>
## 1 Afghanistan           31  
## 2 Angola                40.1
## 3 Albania               52.2
## 4 United Arab Emirates  51.6
## 5 Argentina             47  
## 6 Armenia               44.9
\end{verbatim}

\begin{Shaded}
\begin{Highlighting}[]
\FunctionTok{write\_csv}\NormalTok{(environmental\_quality\_table\_cleaned, }\StringTok{"environmental\_quality\_table\_cleaned.csv"}\NormalTok{)}
\end{Highlighting}
\end{Shaded}

\section{4. Descriptive Statistics and
Plots}\label{descriptive-statistics-and-plots}

\begin{Shaded}
\begin{Highlighting}[]
\CommentTok{\# Clean country names function}
\NormalTok{clean\_country\_names }\OtherTok{\textless{}{-}} \ControlFlowTok{function}\NormalTok{(data, }\AttributeTok{country\_column =} \StringTok{"country"}\NormalTok{) \{}
\NormalTok{  data }\SpecialCharTok{\%\textgreater{}\%}
    \FunctionTok{mutate}\NormalTok{(}\SpecialCharTok{!!}\FunctionTok{sym}\NormalTok{(country\_column) }\SpecialCharTok{:=} \FunctionTok{gsub}\NormalTok{(}\StringTok{"[\^{}[:alnum:] ]"}\NormalTok{, }\StringTok{""}\NormalTok{, }\SpecialCharTok{!!}\FunctionTok{sym}\NormalTok{(country\_column))) }\SpecialCharTok{\%\textgreater{}\%}
    \FunctionTok{mutate}\NormalTok{(}\SpecialCharTok{!!}\FunctionTok{sym}\NormalTok{(country\_column) }\SpecialCharTok{:=} \FunctionTok{trimws}\NormalTok{(}\SpecialCharTok{!!}\FunctionTok{sym}\NormalTok{(country\_column))) }\SpecialCharTok{\%\textgreater{}\%}
    \FunctionTok{mutate}\NormalTok{(}\SpecialCharTok{!!}\FunctionTok{sym}\NormalTok{(country\_column) }\SpecialCharTok{:=} \FunctionTok{tolower}\NormalTok{(}\SpecialCharTok{!!}\FunctionTok{sym}\NormalTok{(country\_column)))}
\NormalTok{\}}

\CommentTok{\# Clean the country names in all datasets}
\NormalTok{happiest\_countries\_cleaned }\OtherTok{\textless{}{-}} \FunctionTok{clean\_country\_names}\NormalTok{(happiest\_countries\_cleaned)}
\NormalTok{gdp\_per\_capita\_table\_cleaned }\OtherTok{\textless{}{-}} \FunctionTok{clean\_country\_names}\NormalTok{(gdp\_per\_capita\_table\_cleaned)}
\NormalTok{literacy\_rate\_table\_cleaned }\OtherTok{\textless{}{-}} \FunctionTok{clean\_country\_names}\NormalTok{(literacy\_rate\_table\_cleaned)}
\NormalTok{environmental\_quality\_table\_cleaned }\OtherTok{\textless{}{-}} \FunctionTok{clean\_country\_names}\NormalTok{(environmental\_quality\_table\_cleaned)}

\CommentTok{\# Rename columns to prepare for merging}
\NormalTok{gdp\_data }\OtherTok{\textless{}{-}}\NormalTok{ gdp\_per\_capita\_table\_cleaned }\SpecialCharTok{\%\textgreater{}\%}
  \FunctionTok{rename}\NormalTok{(}\AttributeTok{Value =}\NormalTok{ estimate) }\SpecialCharTok{\%\textgreater{}\%}
  \FunctionTok{mutate}\NormalTok{(}\AttributeTok{Variable =} \StringTok{"GDP"}\NormalTok{)}

\NormalTok{education\_data }\OtherTok{\textless{}{-}}\NormalTok{ literacy\_rate\_table\_cleaned }\SpecialCharTok{\%\textgreater{}\%}
  \FunctionTok{rename}\NormalTok{(}\AttributeTok{Value =}\NormalTok{ total) }\SpecialCharTok{\%\textgreater{}\%}
  \FunctionTok{mutate}\NormalTok{(}\AttributeTok{Variable =} \StringTok{"Education"}\NormalTok{)}

\NormalTok{env\_quality\_data }\OtherTok{\textless{}{-}}\NormalTok{ environmental\_quality\_table\_cleaned }\SpecialCharTok{\%\textgreater{}\%}
  \FunctionTok{rename}\NormalTok{(}\AttributeTok{Value =}\NormalTok{ value) }\SpecialCharTok{\%\textgreater{}\%}
  \FunctionTok{mutate}\NormalTok{(}\AttributeTok{Variable =} \StringTok{"EnvironmentalQuality"}\NormalTok{)}

\NormalTok{happiness\_data }\OtherTok{\textless{}{-}}\NormalTok{ happiest\_countries\_cleaned }\SpecialCharTok{\%\textgreater{}\%}
  \FunctionTok{rename}\NormalTok{(}\AttributeTok{Value =}\NormalTok{ HappinessScore) }\SpecialCharTok{\%\textgreater{}\%}
  \FunctionTok{mutate}\NormalTok{(}\AttributeTok{Variable =} \StringTok{"HappinessScore"}\NormalTok{)}

\CommentTok{\# Combine all datasets into a single long format}
\NormalTok{long\_data }\OtherTok{\textless{}{-}} \FunctionTok{bind\_rows}\NormalTok{(}
\NormalTok{  gdp\_data,}
\NormalTok{  education\_data,}
\NormalTok{  env\_quality\_data,}
\NormalTok{  happiness\_data}
\NormalTok{)}

\CommentTok{\# Pivot to wide format}
\NormalTok{wide\_data }\OtherTok{\textless{}{-}}\NormalTok{ long\_data }\SpecialCharTok{\%\textgreater{}\%}
  \FunctionTok{pivot\_wider}\NormalTok{(}
    \AttributeTok{names\_from =}\NormalTok{ Variable,}
    \AttributeTok{values\_from =}\NormalTok{ Value}
\NormalTok{  )}

\CommentTok{\# Remove year and sexgap columns}
\NormalTok{wide\_data\_cleaned }\OtherTok{\textless{}{-}}\NormalTok{ wide\_data }\SpecialCharTok{\%\textgreater{}\%}
  \FunctionTok{select}\NormalTok{(}\SpecialCharTok{{-}}\NormalTok{year) }

\CommentTok{\# Display the Merged Data Overview}
\NormalTok{wide\_data\_cleaned }\SpecialCharTok{\%\textgreater{}\%}
  \FunctionTok{head}\NormalTok{(}\DecValTok{10}\NormalTok{) }\SpecialCharTok{\%\textgreater{}\%} 
  \FunctionTok{kable}\NormalTok{(}
    \AttributeTok{col.names =} \FunctionTok{c}\NormalTok{(}\StringTok{"Country"}\NormalTok{, }\StringTok{"GDP"}\NormalTok{, }\StringTok{"Education"}\NormalTok{, }\StringTok{"Environmental Quality"}\NormalTok{, }\StringTok{"Happiness Score"}\NormalTok{),}
    \AttributeTok{caption =} \StringTok{"Merged Data Overview"}\NormalTok{,}
    \AttributeTok{format =} \StringTok{"markdown"}
\NormalTok{  )}
\end{Highlighting}
\end{Shaded}

\begin{longtable}[]{@{}
  >{\raggedright\arraybackslash}p{(\columnwidth - 8\tabcolsep) * \real{0.2143}}
  >{\raggedleft\arraybackslash}p{(\columnwidth - 8\tabcolsep) * \real{0.1000}}
  >{\raggedleft\arraybackslash}p{(\columnwidth - 8\tabcolsep) * \real{0.1429}}
  >{\raggedleft\arraybackslash}p{(\columnwidth - 8\tabcolsep) * \real{0.3143}}
  >{\raggedleft\arraybackslash}p{(\columnwidth - 8\tabcolsep) * \real{0.2286}}@{}}
\caption{Merged Data Overview}\tabularnewline
\toprule\noalign{}
\begin{minipage}[b]{\linewidth}\raggedright
Country
\end{minipage} & \begin{minipage}[b]{\linewidth}\raggedleft
GDP
\end{minipage} & \begin{minipage}[b]{\linewidth}\raggedleft
Education
\end{minipage} & \begin{minipage}[b]{\linewidth}\raggedleft
Environmental Quality
\end{minipage} & \begin{minipage}[b]{\linewidth}\raggedleft
Happiness Score
\end{minipage} \\
\midrule\noalign{}
\endfirsthead
\toprule\noalign{}
\begin{minipage}[b]{\linewidth}\raggedright
Country
\end{minipage} & \begin{minipage}[b]{\linewidth}\raggedleft
GDP
\end{minipage} & \begin{minipage}[b]{\linewidth}\raggedleft
Education
\end{minipage} & \begin{minipage}[b]{\linewidth}\raggedleft
Environmental Quality
\end{minipage} & \begin{minipage}[b]{\linewidth}\raggedleft
Happiness Score
\end{minipage} \\
\midrule\noalign{}
\endhead
\bottomrule\noalign{}
\endlastfoot
luxembourg & 135321 & 100 & 75.1 & 7.12 \\
bermuda & 135321 & NA & NA & NA \\
switzerland & 106098 & 99 & 67.8 & 7.06 \\
ireland & 103500 & 99 & 65.8 & 6.84 \\
cayman islands & 103500 & NA & NA & NA \\
isle of man & 103500 & NA & NA & NA \\
norway & 90434 & 100 & 69.9 & 7.30 \\
singapore & 89370 & NA & 53.0 & 6.52 \\
united states & 86601 & 86 & 57.2 & 6.73 \\
iceland & 85787 & 99 & 64.3 & 7.53 \\
\end{longtable}

I created the Merged Data Overview table to provide a comprehensive view
of all the important variables in a single table. However, due to
differences in country names and variations in the countries included in
each dataset, many NA values are present. Therefore, for the
visualizations, I will analyze two variables at a time separately.
Despite the NA values, this table is still useful for quickly skimming
through the data.

\subsection{RQ1: How does GDP per capita associate with a country's
happiness
score?}\label{rq1-how-does-gdp-per-capita-associate-with-a-countrys-happiness-score}

\begin{Shaded}
\begin{Highlighting}[]
\CommentTok{\# Clean the country names in gdp\_per\_capita\_table\_cleaned}
\NormalTok{gdp\_per\_capita\_table\_cleaned }\OtherTok{\textless{}{-}}\NormalTok{ gdp\_per\_capita\_table\_cleaned }\SpecialCharTok{\%\textgreater{}\%}
  \FunctionTok{mutate}\NormalTok{(}\AttributeTok{country =} \FunctionTok{gsub}\NormalTok{(}\StringTok{"[\^{}[:alnum:] ]"}\NormalTok{, }\StringTok{""}\NormalTok{, country)) }\SpecialCharTok{\%\textgreater{}\%}  
  \FunctionTok{mutate}\NormalTok{(}\AttributeTok{country =} \FunctionTok{trimws}\NormalTok{(country)) }\SpecialCharTok{\%\textgreater{}\%}                   
  \FunctionTok{mutate}\NormalTok{(}\AttributeTok{country =} \FunctionTok{tolower}\NormalTok{(country))                      }

\CommentTok{\# Clean the country names in happiest\_countries\_cleaned}
\NormalTok{happiest\_countries\_cleaned }\OtherTok{\textless{}{-}}\NormalTok{ happiest\_countries\_cleaned }\SpecialCharTok{\%\textgreater{}\%}
  \FunctionTok{mutate}\NormalTok{(}\AttributeTok{country =} \FunctionTok{gsub}\NormalTok{(}\StringTok{"[\^{}[:alnum:] ]"}\NormalTok{, }\StringTok{""}\NormalTok{, country)) }\SpecialCharTok{\%\textgreater{}\%}  
  \FunctionTok{mutate}\NormalTok{(}\AttributeTok{country =} \FunctionTok{trimws}\NormalTok{(country)) }\SpecialCharTok{\%\textgreater{}\%}                    
  \FunctionTok{mutate}\NormalTok{(}\AttributeTok{country =} \FunctionTok{tolower}\NormalTok{(country))                     }

\CommentTok{\# Merge GDP and HappinessScore data}
\NormalTok{merged\_data }\OtherTok{\textless{}{-}}\NormalTok{ gdp\_per\_capita\_table\_cleaned }\SpecialCharTok{\%\textgreater{}\%}
  \FunctionTok{left\_join}\NormalTok{(happiest\_countries\_cleaned, }\AttributeTok{by =} \StringTok{"country"}\NormalTok{)}
\FunctionTok{print}\NormalTok{(merged\_data)}
\end{Highlighting}
\end{Shaded}

\begin{verbatim}
## # A tibble: 221 x 3
##    country        estimate HappinessScore
##    <chr>             <dbl>          <dbl>
##  1 luxembourg       135321           7.12
##  2 bermuda          135321          NA   
##  3 switzerland      106098           7.06
##  4 ireland          103500           6.84
##  5 cayman islands   103500          NA   
##  6 isle of man      103500          NA   
##  7 norway            90434           7.3 
##  8 singapore         89370           6.52
##  9 united states     86601           6.73
## 10 iceland           85787           7.53
## # i 211 more rows
\end{verbatim}

\begin{Shaded}
\begin{Highlighting}[]
\CommentTok{\# Remove missing values}
\NormalTok{cleaned\_merged\_data }\OtherTok{\textless{}{-}}\NormalTok{ merged\_data }\SpecialCharTok{\%\textgreater{}\%}
  \FunctionTok{filter}\NormalTok{(}\SpecialCharTok{!}\FunctionTok{is.na}\NormalTok{(estimate), }\SpecialCharTok{!}\FunctionTok{is.na}\NormalTok{(HappinessScore))}
\FunctionTok{print}\NormalTok{(cleaned\_merged\_data)}
\end{Highlighting}
\end{Shaded}

\begin{verbatim}
## # A tibble: 140 x 3
##    country       estimate HappinessScore
##    <chr>            <dbl>          <dbl>
##  1 luxembourg      135321           7.12
##  2 switzerland     106098           7.06
##  3 ireland         103500           6.84
##  4 norway           90434           7.3 
##  5 singapore        89370           6.52
##  6 united states    86601           6.73
##  7 iceland          85787           7.53
##  8 denmark          69273           7.58
##  9 netherlands      67984           7.32
## 10 australia        65966           7.06
## # i 130 more rows
\end{verbatim}

\begin{Shaded}
\begin{Highlighting}[]
\CommentTok{\# View summary statistics }
\FunctionTok{summary}\NormalTok{(cleaned\_merged\_data[, }\FunctionTok{c}\NormalTok{(}\StringTok{"estimate"}\NormalTok{, }\StringTok{"HappinessScore"}\NormalTok{)])}
\end{Highlighting}
\end{Shaded}

\begin{verbatim}
##     estimate      HappinessScore 
##  Min.   :   411   Min.   :1.720  
##  1st Qu.:  2680   1st Qu.:4.635  
##  Median :  7507   Median :5.805  
##  Mean   : 19722   Mean   :5.522  
##  3rd Qu.: 29399   3rd Qu.:6.412  
##  Max.   :135321   Max.   :7.740
\end{verbatim}

\begin{Shaded}
\begin{Highlighting}[]
\CommentTok{\# Create the scatter plot}
\FunctionTok{ggplot}\NormalTok{(}\AttributeTok{data =}\NormalTok{ cleaned\_merged\_data, }\FunctionTok{aes}\NormalTok{(}\AttributeTok{x =}\NormalTok{ estimate, }\AttributeTok{y =}\NormalTok{ HappinessScore)) }\SpecialCharTok{+}
    \FunctionTok{geom\_point}\NormalTok{(}\AttributeTok{color =} \StringTok{"blue"}\NormalTok{, }\AttributeTok{size =} \DecValTok{2}\NormalTok{) }\SpecialCharTok{+}                
    \FunctionTok{geom\_smooth}\NormalTok{(}\AttributeTok{method =} \StringTok{"lm"}\NormalTok{, }\AttributeTok{color =} \StringTok{"red"}\NormalTok{, }\AttributeTok{se =} \ConstantTok{TRUE}\NormalTok{) }\SpecialCharTok{+} 
    \FunctionTok{scale\_x\_continuous}\NormalTok{(}\AttributeTok{labels =}\NormalTok{ scales}\SpecialCharTok{::}\NormalTok{comma) }\SpecialCharTok{+}          
    \FunctionTok{labs}\NormalTok{(}
        \AttributeTok{x =} \StringTok{"GDP Per Capita (USD)"}\NormalTok{,}
        \AttributeTok{y =} \StringTok{"Happiness Score"}\NormalTok{,}
        \AttributeTok{title =} \StringTok{"GDP Per Capita and Happiness Score"}
\NormalTok{    ) }\SpecialCharTok{+}
  \FunctionTok{theme\_minimal}\NormalTok{() }\SpecialCharTok{+} 
  \FunctionTok{theme}\NormalTok{(}
    \AttributeTok{plot.title =} \FunctionTok{element\_text}\NormalTok{(}\AttributeTok{size =} \DecValTok{12}\NormalTok{) }
\NormalTok{  )}
\end{Highlighting}
\end{Shaded}

\begin{verbatim}
## `geom_smooth()` using formula = 'y ~ x'
\end{verbatim}

\includegraphics{Final-Report_files/figure-latex/unnamed-chunk-10-1.pdf}

\subsection{RQ2: How does environmental quality associate with a
country's happiness
score?}\label{rq2-how-does-environmental-quality-associate-with-a-countrys-happiness-score}

\begin{Shaded}
\begin{Highlighting}[]
\CommentTok{\# Clean the country names in environmental\_quality\_table\_cleaned}
\NormalTok{environmental\_quality\_table\_cleaned }\OtherTok{\textless{}{-}}\NormalTok{ environmental\_quality\_table\_cleaned }\SpecialCharTok{\%\textgreater{}\%}
  \FunctionTok{mutate}\NormalTok{(}\AttributeTok{country =} \FunctionTok{gsub}\NormalTok{(}\StringTok{"[\^{}[:alnum:] ]"}\NormalTok{, }\StringTok{""}\NormalTok{, country)) }\SpecialCharTok{\%\textgreater{}\%}  
  \FunctionTok{mutate}\NormalTok{(}\AttributeTok{country =} \FunctionTok{trimws}\NormalTok{(country)) }\SpecialCharTok{\%\textgreater{}\%}                  
  \FunctionTok{mutate}\NormalTok{(}\AttributeTok{country =} \FunctionTok{tolower}\NormalTok{(country))                     }

\CommentTok{\# Clean the country names in happiest\_countries\_cleaned}
\NormalTok{happiest\_countries\_cleaned }\OtherTok{\textless{}{-}}\NormalTok{ happiest\_countries\_cleaned }\SpecialCharTok{\%\textgreater{}\%}
  \FunctionTok{mutate}\NormalTok{(}\AttributeTok{country =} \FunctionTok{gsub}\NormalTok{(}\StringTok{"[\^{}[:alnum:] ]"}\NormalTok{, }\StringTok{""}\NormalTok{, country)) }\SpecialCharTok{\%\textgreater{}\%}  
  \FunctionTok{mutate}\NormalTok{(}\AttributeTok{country =} \FunctionTok{trimws}\NormalTok{(country)) }\SpecialCharTok{\%\textgreater{}\%}                    
  \FunctionTok{mutate}\NormalTok{(}\AttributeTok{country =} \FunctionTok{tolower}\NormalTok{(country))                     }

\CommentTok{\# Merge environmental quality data with happiness data}
\NormalTok{merged\_data\_env }\OtherTok{\textless{}{-}}\NormalTok{ environmental\_quality\_table\_cleaned }\SpecialCharTok{\%\textgreater{}\%}
  \FunctionTok{left\_join}\NormalTok{(happiest\_countries\_cleaned, }\AttributeTok{by =} \StringTok{"country"}\NormalTok{)}
\FunctionTok{print}\NormalTok{(merged\_data\_env)}
\end{Highlighting}
\end{Shaded}

\begin{verbatim}
## # A tibble: 179 x 3
##    country              value HappinessScore
##    <chr>                <dbl>          <dbl>
##  1 afghanistan           31             1.72
##  2 angola                40.1          NA   
##  3 albania               52.2           5.3 
##  4 united arab emirates  51.6           6.73
##  5 argentina             47             6.19
##  6 armenia               44.9           5.46
##  7 antigua and barbuda   55.6          NA   
##  8 australia             63.1           7.06
##  9 austria               68.9           6.91
## 10 azerbaijan            40.5           4.89
## # i 169 more rows
\end{verbatim}

\begin{Shaded}
\begin{Highlighting}[]
\CommentTok{\# Remove rows with missing values in value or HappinessScore}
\NormalTok{cleaned\_merged\_data\_env }\OtherTok{\textless{}{-}}\NormalTok{ merged\_data\_env }\SpecialCharTok{\%\textgreater{}\%}
  \FunctionTok{filter}\NormalTok{(}\SpecialCharTok{!}\FunctionTok{is.na}\NormalTok{(value), }\SpecialCharTok{!}\FunctionTok{is.na}\NormalTok{(HappinessScore))}
\FunctionTok{print}\NormalTok{(cleaned\_merged\_data\_env)}
\end{Highlighting}
\end{Shaded}

\begin{verbatim}
## # A tibble: 135 x 3
##    country              value HappinessScore
##    <chr>                <dbl>          <dbl>
##  1 afghanistan           31             1.72
##  2 albania               52.2           5.3 
##  3 united arab emirates  51.6           6.73
##  4 argentina             47             6.19
##  5 armenia               44.9           5.46
##  6 australia             63.1           7.06
##  7 austria               68.9           6.91
##  8 azerbaijan            40.5           4.89
##  9 belgium               66.8           6.89
## 10 benin                 37.8           4.38
## # i 125 more rows
\end{verbatim}

\begin{Shaded}
\begin{Highlighting}[]
\CommentTok{\# View summary statistics }
\FunctionTok{summary}\NormalTok{(cleaned\_merged\_data\_env[, }\FunctionTok{c}\NormalTok{(}\StringTok{"value"}\NormalTok{)])}
\end{Highlighting}
\end{Shaded}

\begin{verbatim}
##      value      
##  Min.   :24.60  
##  1st Qu.:38.40  
##  Median :46.10  
##  Mean   :47.83  
##  3rd Qu.:56.70  
##  Max.   :75.70
\end{verbatim}

\begin{Shaded}
\begin{Highlighting}[]
\CommentTok{\# Create the line graph}
\FunctionTok{ggplot}\NormalTok{(cleaned\_merged\_data\_env, }\FunctionTok{aes}\NormalTok{(}\AttributeTok{x =}\NormalTok{ value, }\AttributeTok{y =}\NormalTok{ HappinessScore)) }\SpecialCharTok{+}
  \FunctionTok{geom\_line}\NormalTok{(}\AttributeTok{color =} \StringTok{"blue"}\NormalTok{, }\AttributeTok{size =} \DecValTok{1}\NormalTok{) }\SpecialCharTok{+}  
  \FunctionTok{geom\_point}\NormalTok{(}\AttributeTok{color =} \StringTok{"red"}\NormalTok{, }\AttributeTok{size =} \DecValTok{2}\NormalTok{) }\SpecialCharTok{+} 
  \FunctionTok{labs}\NormalTok{(}
    \AttributeTok{title =} \StringTok{"Environmental Index and Happiness Score"}\NormalTok{,}
    \AttributeTok{x =} \StringTok{"Environmental Index"}\NormalTok{,}
    \AttributeTok{y =} \StringTok{"Happiness Score"}
\NormalTok{  ) }\SpecialCharTok{+}
  \FunctionTok{theme\_minimal}\NormalTok{() }\SpecialCharTok{+} 
  \FunctionTok{theme}\NormalTok{(}
    \AttributeTok{plot.title =} \FunctionTok{element\_text}\NormalTok{(}\AttributeTok{size =} \DecValTok{12}\NormalTok{) }
\NormalTok{  )}
\end{Highlighting}
\end{Shaded}

\begin{verbatim}
## Warning: Using `size` aesthetic for lines was deprecated in ggplot2 3.4.0.
## i Please use `linewidth` instead.
## This warning is displayed once every 8 hours.
## Call `lifecycle::last_lifecycle_warnings()` to see where this warning was
## generated.
\end{verbatim}

\includegraphics{Final-Report_files/figure-latex/unnamed-chunk-11-1.pdf}

\begin{Shaded}
\begin{Highlighting}[]
\CommentTok{\# Create the scatter plot}
\FunctionTok{ggplot}\NormalTok{(cleaned\_merged\_data\_env, }\FunctionTok{aes}\NormalTok{(}\AttributeTok{x =}\NormalTok{ value, }\AttributeTok{y =}\NormalTok{ HappinessScore)) }\SpecialCharTok{+}
  \FunctionTok{geom\_point}\NormalTok{(}\AttributeTok{color =} \StringTok{"blue"}\NormalTok{, }\AttributeTok{size =} \DecValTok{2}\NormalTok{) }\SpecialCharTok{+}               
  \FunctionTok{geom\_smooth}\NormalTok{(}\AttributeTok{method =} \StringTok{"lm"}\NormalTok{, }\AttributeTok{color =} \StringTok{"red"}\NormalTok{, }\AttributeTok{se =} \ConstantTok{TRUE}\NormalTok{) }\SpecialCharTok{+} 
  \FunctionTok{labs}\NormalTok{(}
    \AttributeTok{title =} \StringTok{"Environmental Index and Happiness Score"}\NormalTok{,}
    \AttributeTok{x =} \StringTok{"Environmental Index"}\NormalTok{,}
    \AttributeTok{y =} \StringTok{"Happiness Score"}
\NormalTok{  ) }\SpecialCharTok{+}
  \FunctionTok{theme\_minimal}\NormalTok{() }\SpecialCharTok{+} 
  \FunctionTok{theme}\NormalTok{(}
    \AttributeTok{plot.title =} \FunctionTok{element\_text}\NormalTok{(}\AttributeTok{size =} \DecValTok{12}\NormalTok{) }
\NormalTok{  )}
\end{Highlighting}
\end{Shaded}

\begin{verbatim}
## `geom_smooth()` using formula = 'y ~ x'
\end{verbatim}

\includegraphics{Final-Report_files/figure-latex/unnamed-chunk-11-2.pdf}

I created two graphs to explore and to find out the best graph. The line
graph still helps demonstrate a general positive trend between
environmental quality and happiness scores; however, I think the scatter
plot graph is better.

\subsection{RQ3: How does a country's literacy rate associate with its
happiness
score?}\label{rq3-how-does-a-countrys-literacy-rate-associate-with-its-happiness-score}

\begin{Shaded}
\begin{Highlighting}[]
\CommentTok{\# Clean the country names in literacy\_rate\_table\_cleaned}
\NormalTok{literacy\_rate\_table\_cleaned }\OtherTok{\textless{}{-}}\NormalTok{ literacy\_rate\_table\_cleaned }\SpecialCharTok{\%\textgreater{}\%}
  \FunctionTok{mutate}\NormalTok{(}\AttributeTok{country =} \FunctionTok{gsub}\NormalTok{(}\StringTok{"[\^{}[:alnum:] ]"}\NormalTok{, }\StringTok{""}\NormalTok{, country)) }\SpecialCharTok{\%\textgreater{}\%} 
  \FunctionTok{mutate}\NormalTok{(}\AttributeTok{country =} \FunctionTok{trimws}\NormalTok{(country)) }\SpecialCharTok{\%\textgreater{}\%}                    
  \FunctionTok{mutate}\NormalTok{(}\AttributeTok{country =} \FunctionTok{tolower}\NormalTok{(country))                    }

\CommentTok{\# Clean the country names in happiest\_countries\_cleaned}
\NormalTok{happiest\_countries\_cleaned }\OtherTok{\textless{}{-}}\NormalTok{ happiest\_countries\_cleaned }\SpecialCharTok{\%\textgreater{}\%}
  \FunctionTok{mutate}\NormalTok{(}\AttributeTok{country =} \FunctionTok{gsub}\NormalTok{(}\StringTok{"[\^{}[:alnum:] ]"}\NormalTok{, }\StringTok{""}\NormalTok{, country)) }\SpecialCharTok{\%\textgreater{}\%} 
  \FunctionTok{mutate}\NormalTok{(}\AttributeTok{country =} \FunctionTok{trimws}\NormalTok{(country)) }\SpecialCharTok{\%\textgreater{}\%}                   
  \FunctionTok{mutate}\NormalTok{(}\AttributeTok{country =} \FunctionTok{tolower}\NormalTok{(country))                    }

\CommentTok{\# Merge Literacy Rate and HappinessScore data}
\NormalTok{merged\_data\_lit }\OtherTok{\textless{}{-}}\NormalTok{ literacy\_rate\_table\_cleaned }\SpecialCharTok{\%\textgreater{}\%}
  \FunctionTok{left\_join}\NormalTok{(happiest\_countries\_cleaned, }\AttributeTok{by =} \StringTok{"country"}\NormalTok{)}
\FunctionTok{print}\NormalTok{(merged\_data\_lit)}
\end{Highlighting}
\end{Shaded}

\begin{verbatim}
## # A tibble: 195 x 4
##    country             total  year HappinessScore
##    <chr>               <dbl> <dbl>          <dbl>
##  1 afghanistan          37.3  2021           1.72
##  2 albania              98.1  2018           5.3 
##  3 algeria              81.4  2018           5.36
##  4 andorra             100    2016          NA   
##  5 angola               71.1  2015          NA   
##  6 antigua and barbuda  99    2015          NA   
##  7 argentina            99    2018           6.19
##  8 armenia              99.8  2020           5.46
##  9 australia            99      NA           7.06
## 10 austria              98      NA           6.91
## # i 185 more rows
\end{verbatim}

\begin{Shaded}
\begin{Highlighting}[]
\CommentTok{\# Remove missing values}
\NormalTok{cleaned\_merged\_data\_lit }\OtherTok{\textless{}{-}}\NormalTok{ merged\_data\_lit }\SpecialCharTok{\%\textgreater{}\%}
  \FunctionTok{filter}\NormalTok{(}\SpecialCharTok{!}\FunctionTok{is.na}\NormalTok{(total), }\SpecialCharTok{!}\FunctionTok{is.na}\NormalTok{(year), }\SpecialCharTok{!}\FunctionTok{is.na}\NormalTok{(HappinessScore))}
\FunctionTok{print}\NormalTok{(cleaned\_merged\_data\_lit)}
\end{Highlighting}
\end{Shaded}

\begin{verbatim}
## # A tibble: 116 x 4
##    country                total  year HappinessScore
##    <chr>                  <dbl> <dbl>          <dbl>
##  1 afghanistan             37.3  2021           1.72
##  2 albania                 98.1  2018           5.3 
##  3 algeria                 81.4  2018           5.36
##  4 argentina               99    2018           6.19
##  5 armenia                 99.8  2020           5.46
##  6 azerbaijan              99.8  2019           4.89
##  7 bahrain                 97.5  2018           5.96
##  8 benin                   42.4  2018           4.38
##  9 bolivia                 92.5  2015           5.78
## 10 bosnia and herzegovina  98.5  2015           5.88
## # i 106 more rows
\end{verbatim}

\begin{Shaded}
\begin{Highlighting}[]
\CommentTok{\# View summary statistics }
\FunctionTok{summary}\NormalTok{(cleaned\_merged\_data\_lit[, }\FunctionTok{c}\NormalTok{(}\StringTok{"total"}\NormalTok{)])}
\end{Highlighting}
\end{Shaded}

\begin{verbatim}
##      total       
##  Min.   : 22.30  
##  1st Qu.: 78.72  
##  Median : 93.80  
##  Mean   : 85.00  
##  3rd Qu.: 98.40  
##  Max.   :100.00
\end{verbatim}

\begin{Shaded}
\begin{Highlighting}[]
\CommentTok{\# Calculate percentiles for the "total" column in the literacy rate table}
\NormalTok{percentiles }\OtherTok{\textless{}{-}} \FunctionTok{quantile}\NormalTok{(literacy\_rate\_table\_cleaned}\SpecialCharTok{$}\NormalTok{total, }\AttributeTok{probs =} \FunctionTok{c}\NormalTok{(}\FloatTok{0.25}\NormalTok{, }\FloatTok{0.5}\NormalTok{, }\FloatTok{0.75}\NormalTok{, }\DecValTok{1}\NormalTok{), }\AttributeTok{na.rm =} \ConstantTok{TRUE}\NormalTok{)}
\FunctionTok{print}\NormalTok{(percentiles)}
\end{Highlighting}
\end{Shaded}

\begin{verbatim}
##     25%     50%     75%    100% 
##  80.725  95.650  99.000 100.000
\end{verbatim}

\begin{Shaded}
\begin{Highlighting}[]
\CommentTok{\# Add bins to the data using the specified quartiles}
\NormalTok{cleaned\_merged\_data\_lit }\OtherTok{\textless{}{-}}\NormalTok{ cleaned\_merged\_data\_lit }\SpecialCharTok{\%\textgreater{}\%}
  \FunctionTok{mutate}\NormalTok{(}
    \AttributeTok{literacy\_bin =} \FunctionTok{cut}\NormalTok{(}
\NormalTok{      total,}
      \AttributeTok{breaks =} \FunctionTok{c}\NormalTok{(}\DecValTok{0}\NormalTok{, }\FloatTok{80.725}\NormalTok{, }\FloatTok{95.650}\NormalTok{, }\FloatTok{99.000}\NormalTok{, }\DecValTok{100}\NormalTok{), }
      \AttributeTok{labels =} \FunctionTok{c}\NormalTok{(}\StringTok{"Q1: \textless{}80.725"}\NormalTok{, }
                 \StringTok{"Q2: 80.725{-}95.650"}\NormalTok{, }
                 \StringTok{"Q3: 95.650{-}99.000"}\NormalTok{, }
                 \StringTok{"Q4: \textgreater{}99.000"}\NormalTok{),}
      \AttributeTok{include.lowest =} \ConstantTok{TRUE}
\NormalTok{    )}
\NormalTok{  )}
\CommentTok{\# Create the box plot}
\FunctionTok{ggplot}\NormalTok{(cleaned\_merged\_data\_lit, }\FunctionTok{aes}\NormalTok{(}\AttributeTok{x =}\NormalTok{ literacy\_bin, }\AttributeTok{y =}\NormalTok{ HappinessScore)) }\SpecialCharTok{+}
  \FunctionTok{geom\_boxplot}\NormalTok{(}\AttributeTok{fill =} \StringTok{"lightblue"}\NormalTok{, }\AttributeTok{color =} \StringTok{"black"}\NormalTok{, }\AttributeTok{outlier.color =} \StringTok{"red"}\NormalTok{, }\AttributeTok{outlier.size =} \DecValTok{2}\NormalTok{) }\SpecialCharTok{+}
  \FunctionTok{labs}\NormalTok{(}
    \AttributeTok{title =} \StringTok{"Literacy Rate and Happiness Score"}\NormalTok{,}
    \AttributeTok{x =} \StringTok{"Literacy Rate (Quartiles)"}\NormalTok{,}
    \AttributeTok{y =} \StringTok{"Happiness Score"}
\NormalTok{  ) }\SpecialCharTok{+}
  \FunctionTok{theme\_minimal}\NormalTok{() }\SpecialCharTok{+} 
  \FunctionTok{theme}\NormalTok{(}
    \AttributeTok{plot.title =} \FunctionTok{element\_text}\NormalTok{(}\AttributeTok{size =} \DecValTok{12}\NormalTok{) }
\NormalTok{  )}
\end{Highlighting}
\end{Shaded}

\includegraphics{Final-Report_files/figure-latex/unnamed-chunk-12-1.pdf}
\# 5. Data Interpretation \{\#data-interpretation\}

\subsection{Data Analysis}\label{data-analysis}

In addressing the main question, ``How do economic, environmental, and
educational variables influence national happiness levels as reported by
the World Population Review?'', this analysis provides evidence of
significant associations between these factors and happiness scores.

\emph{H1: I hypothesize that higher GDP per capita, measured in USD
based on IMF estimates, is associated with higher national happiness
levels.}

The scatter plot examining GDP per capita and happiness scores reveals a
clear positive trend. Countries with higher GDP per capita tend to
report higher happiness scores. This supports the hypothesis that
wealthier countries have more resources to invest in public services
such as healthcare, education, and infrastructure, which improve the
overall quality of life.

\emph{H2: I hypothesize that better environmental quality, as measured
by the Environmental Performance Index (EPI), is associated with higher
happiness.}

The line graph analyzing the association between environmental quality
and happiness scores suggests a positive but variable association.
Countries with higher environmental index scores generally report higher
happiness levels, supporting the hypothesis that a clean and sustainable
environment promotes well-being. However, the noisy pattern in the graph
indicates that some countries with similar environmental scores differ
significantly in happiness. The scatter plot graph shows a positive
trend as well.

\emph{H3: I hypothesize that higher literacy rates, as a measure of
education, are associated with higher happiness scores.}

The box plot demonstrates a generally positive association between
literacy rates and happiness scores across the first three quartiles (Q1
to Q3). However, Q3 (95.650--99.000) appears to have a slightly higher
happiness scores compared to Q4 (\textgreater{} 99.000), despite Q4
representing the countries with the highest literacy rates. This pattern
could reflect diminishing returns of literacy on happiness or challenges
unique to highly literate societies, such as increased societal
pressures, disparities in resource allocation, or inequality.

\subsection{Limitations and Future
Directions}\label{limitations-and-future-directions}

One limitation of this study lies in the quality and consistency of the
data used. Data sources may vary in their methods of collection,
measurements, and reporting standards across countries. Future research
should prioritize using standardized global datasets to minimize
inconsistencies and align data collection periods across variables to
ensure accuracy. Conducting longitudinal studies could further enhance
the understanding of how changes in variables influence happiness over
time, providing a more comprehensive and precise perspective on these
relationships.

As an extension of this study, we could perform statistical correlation
tests, such as Pearson or Spearman correlation, to quantitatively assess
the strength and direction of relationships between each variable and
national happiness scores. This approach would allow us to identify
which factor is most strongly correlated with happiness and could
potentially be a primary determinant. Additionally, conducting
regression analysis could help control for confounding variables and
provide deeper insights into the relative importance of each factor in
explaining variations in happiness.

\subsection{Implications}\label{implications}

The positive link between GDP per capita and happiness emphasizes the
importance of investments in public services like healthcare, education,
and infrastructure, with a focus on reducing income inequality to
maximize benefits. Similarly, the association between environmental
quality and happiness underscores the need for sustainable policies that
prioritize pollution reduction and green initiatives. The relationship
between literacy rates and happiness suggests that education policies
should focus not only on improving literacy but also on ensuring
equitable access and opportunities, as the trend indicates that higher
literacy rates do not always directly translate to higher happiness
scores. Overall, these findings demonstrate the need for balanced
strategies addressing economic, educational, and environmental factors
to achieve sustainable well-being globally.

\end{document}
